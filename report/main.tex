\documentclass{article}

\usepackage{amsmath, amsfonts, graphicx}
\usepackage{color, listings}
\usepackage[utf8]{inputenc}
\usepackage{CJKutf8}
\usepackage{geometry}
\geometry {
    a4paper,
    left = 3cm,
    right = 3cm,
    top = 4cm,
    bottom = 4cm
}

\lstset {
    backgroundcolor = \color{mygray},
    basicstyle=\ttfamily,
}

\definecolor{mygray}{rgb}{0.92, 0.92, 0.92}
\definecolor{vgreen}{RGB}{104,180,104}
\definecolor{vblue}{RGB}{49,49,255}
\definecolor{vorange}{RGB}{255,143,102}
\lstdefinestyle{verilog-style}
{
    language=Verilog,
    basicstyle=\small,
    keywordstyle=\color{vblue},
    identifierstyle=\color{black},
    commentstyle=\color{vgreen},
    tabsize=8
}

\title{DIC Final Project Report}
\author{通訊四 407430013 林子翔}
\date{\today}

\usepackage{indentfirst}

\begin{document}
    \begin{titlepage}
        \begin{CJK}{UTF8}{bsmi}
        \maketitle
        \end{CJK}
        \thispagestyle{empty}
    \end{titlepage}
    
    \section*{Intoduction}
        In this project, we want to design an floating-point number multiplier chip, 
        which complies with the IEEE-754 standard on 64-bit basic double format with "rounding to nearest" mode. 

        At the first section, I will describe my design on the algorithm, 
        and then I will explain the method to check my verilog program. 
        At the third section, I will show the implementation result and the simulation statistics.


    \section{Algorithm}
        Our multiplier circuit can be simply divided into three stages:
        \begin{enumerate}
            \item Input stage
            \item Calculation stage
            \item Output stage
        \end{enumerate}
        Since the input stage and output stage just use a counter to move in or move out the data, we will mainly focus on how the calculation works.
        \newline\newline
        To describe my code, some used variables are listed here.
        \begin{lstlisting}[style=verilog-style]
    reg [63:0] A, B;    // input data
    reg subnormal;      // subnormal number indicator
    reg sign;
    reg signed [12:0] expn;
    reg [105:0] mprod;
        \end{lstlisting}
        At the end, \{{\tt sign, expn[10:0], mprod[103:52]}\} would be output.

        \begin{itemize}
            \item[$*$] RTL Code Location: \verb|/misc/Si2_RAID-1/COURSE/dic/dic03/final_project/fp_mult.v|
        \end{itemize}

        \subsection{Calculation}
            To clearly state the progress of the algorithm, I am going to describe the calculation cycle-by-cycle.
            \subsubsection*{Clock cycle 0}
                At this time, the input data have located in {\tt A} and {\tt B}. \\
                In the first cycle in calculation stage, we have to check whether {\tt A} or {\tt B} represents some special value, such as NaN and infinity. \\
                If we are in one of the following conditions, raise the {\tt calend} flag and assign {\tt sign}, {\tt expn}, {\tt mprod} to corresponding value. As the flag raised, it would enter the output stage at next cycle, instead of going to the {\bf Clock cycle 1} we would mention later.
                \begin{lstlisting}[style=verilog-style]
    // A is NaN
    if (A[62:52] == {11{1'b1}} && A[51:0])
        calend <= 1;
    // B is NaN
    else if (B[62:52] == {11{1'b1}} && B[51:0])
        calend <= 1;
    // A is 0 and B is infty
    else if (!A[62:0] && B[62:52] == {11{1'b1}} && !B[51:0])
        calend <= 1;
    // B is 0 and A is infty
    else if (!B[62:0] && A[62:52] == {11{1'b1}} && !A[51:0])
        calend <= 1;
    // A or B is 0 and both not infty
    else if (!A[62:0] || !B[62:0])
        calend <= 1;
    // A and B are both subnormal numbers
    else if (!A[62:52] && A[51:0] && !B[62:52] && B[51:0])
        calend <= 1;
                \end{lstlisting}
                
                Simultaneously at cycle 0, if {\tt A} represents a subnormal number, {\tt A} and {\tt B} would be swapped. 
                With this trick, the subnormal number would be always fixed at {\tt B}.
                Afterwards, if we see {\tt subnormal} flag is raised, we would know that {\tt A} is a normal number and {\tt B} is a subnormal number, without checking again.

            \subsubsection*{Clock cycle 1 - 4}
                If we did not leave calculation stage at {\bf clock cycle 0}, there are two possibilities:
                \begin{itemize}
                    \item {\tt A} and {\tt B} are both normal numbers if ({\tt !subnormal}).
                    \item {\tt A} is normal and {\tt B} is subnormal if ({\tt subnormal}).
                \end{itemize}
                When it comes to multiply two floating number, there are three basic operations we need to do:
                \begin{itemize}
                    \item xor their {\bf sign bit}
                    \item add their {\bf exponential part}
                    \item multiply their {\bf fractional part}
                \end{itemize}

                During the four cycles, we want to complete the multiplication part.
                But since they have 52 or 52+1 bits (+1 for the implicit leading 1 of normal numbers),
                it is a huge size for binary mutiplication.
                To prevent overloading one single cycle, we divide this computation to four cycles, as follow:
                \begin{lstlisting}[style = verilog-style]
    else if (inend && !calend && counter == 4'b0001) begin  // 1
        mprod <= {1'b1, A[51:0]} * B[13:0];
    end
    else if (inend && !calend && counter == 4'b0011) begin  // 2
        mprod <= mprod + (({1'b1, A[51:0]} * B[26:14]) << 14);
    end
    else if (inend && !calend && counter == 4'b0010) begin  // 3
        mprod <= mprod + (({1'b1, A[51:0]} * B[39:27]) << 27);
    end
    else if (inend && !calend && counter == 4'b0110) begin  // 4
        mprod <= mprod + (({1'b1, A[51:0]} * 
                                {~subnormal, B[51:40]}) << 40);
    end
                \end{lstlisting}
                \noindent\rule{\textwidth}{0.1pt} \\

                If ({\tt subnormal}), we need to do an additional thing during these four cycles. That is, indexing the highest {\tt 1} in the fractional part of {\tt B}.
                We want a result like this:
                \begin{lstlisting}
    The position of highest 1 in B[51:0]:
    ####################################################
    |                                                  |
    |                                                  |
    1                    - result -                   52
                \end{lstlisting}
                The kind of indexing is a little bit compilicated in verilog HDL. 
                To achieve this, I try to do binary search at first and then apply
                bitwise searching.
                At the beginning, we need to introduce some additional registers:
                \begin{lstlisting}[style = verilog-style]
    reg [5:0] idxMsb;
    reg [2:0] msb_at_block;
    reg [25:0] tmpbuf;  // temp buffer
                \end{lstlisting}
                For example, assume that we have such {\tt B[51:0]}:
                \begin{lstlisting}
    0000000001000_0000000000000_0000000000000_0000000000000
             |
             9
                \end{lstlisting}
                At cycle 1, {\tt B[51:0]} is compared with {(\tt 1 << 26)}, 
                and {\tt msb\_at\_block[2]} is set since {\tt B[51:0] >= (1 << 26)}.
                \begin{lstlisting}
    msb_at_block:
    1xx
    tmpbuf:
    0000000001000_0000000000000
    [                         ] <- next range to compare
                \end{lstlisting}
                Because it is not easy to compare with non-specific part of a number,
                we copy the higher part of the number to its lower part after each comparison, if needed.
                Then we may always do the comparison on the lowest part of {\tt tmpbuf}. \\
                After cycle 2, we do assignment {\tt tmpbuf[12:0] <= tmpbuf[25:13]} since {\tt (tmpbuf[25:0] >= (1 << 13))} and results in:
                \begin{lstlisting}
    msb_at_block:
    11x
    tmpbuf:
    0000000001000_0000000001000
                  [           ] <- next range to compare
                \end{lstlisting}
                At cycle 3, no copy operation involved since {\tt (tmpbuf[12:0] < (1 << 7))}
                \begin{lstlisting}
    msb_at_block:
    110
    tmpbuf:
    0000000001000_0000000001000
                        [     ] <- next range to compare
                \end{lstlisting}
                At cycle 4, we search the range bit-by bit to find the correct position of the highest {\tt 1} in the original {\tt B[51:0]} and store the result to {\tt idxMsb}.

            \subsubsection*{Clock cycle 5}
                Overhere, we have to explain why we computed {\tt idxMsb} at {\bf clock cycle 1-4}.
                When we want to generate the output for our {\tt A}$\times${\tt B} product, 
                we have to find the highest {\tt 1} in {\tt mprod} and hide it if the product is a normal number.
                In the case that {\tt A} and {\tt B} are both normal number, it is very easy because
                \[
                    1.\underbrace{xx\dotsc xx}_{52} \quad\times\quad 1.\underbrace{xx\dotsc xx}_{52}
                \]
                always leads to one of the three:
                \begin{align}
                    01.\underbrace{xx\dotsc xx}_{104} \label{01}\\
                    10.\underbrace{xx\dotsc xx}_{104} \label{10}\\
                    11.\underbrace{xx\dotsc xx}_{104} \label{11}
                \end{align}

                However, on the other hand in the subnormal case, we also need to hide the highest {\tt 1} (if the product is normal) but we do not know where it is.
                Thus we were finding the highest {\tt 1} in {\tt B[51:0]}, and this is nearly equivalent to search in {\tt mprod}, which was not available at that time.

                In this cycle, we only do the alignment for subnormal case:
                \begin{lstlisting}[style = verilog-style]
    if (subnormal)
        mprod <= mprod << idxMsb;
                \end{lstlisting}

                If {\tt (!subnormal)}, do nothing.
                
            \subsubsection*{Clock cycle 6}
                After clock 5, no matter the input contained subnormal number or not. The {\tt mprod} is guaranteed to have the form in Equation \eqref{01}, \eqref{10}, or \eqref{11}. \\
                Then we need to check {\tt mprod[105]}. If it is {\tt 1}, this means the multiplication produced a carry-out. 
                We do 1-bit right-shift on {\tt mprod} to ensure {\tt (mprod[104] == 1)}. 

                Another big thing for this cycle is the computation of {\tt expn}.
                \begin{lstlisting}[style = verilog-style]
    if (subnormal)
        expn <= sign_Aexpn - 11'd1022 - sign_idxMsb + sign_carry;
    else
        expn <= sign_Aexpn + sign_Bexpn - 11'd1023 + sign_carry;
                \end{lstlisting}
                where the variables with "sign" prefix are just the variables with "signed" declaration to execute signed arithmetic operations.

            \subsubsection*{Clock cycle 7}
                According to the value of {\tt expn}, the output have four possibly formats:
                \begin{itemize}
                    \item {\tt expn} $\geq$ {\tt 11'b111\_1111\_1111} \\
                    Since it exceeds the maximum range of valid {\tt expn}, it should be rouned to $\infty$ according to the standard.
                    \item {\tt 11'b111\_1111\_1111} $>$ {\tt expn} $>$ 0 \\
                    Result in a normal number.
                    \item 0 $\geq$ {\tt expn} $\geq$ -52 \\
                    Results in a subnormal number.
                    \item {\tt -52} $>$ {\tt expn} \\
                    Since the value is too tiny, it should be rounded to 0 according to the standard.
                \end{itemize}

                In this cycle, {\tt mprod} is right-shifted to prepare for the subnormal output.
                \begin{lstlisting}[style = verilog-style]
    if (sign_zero >= expn && expn >= -52)
        mprod <= mprod >> (2 + ~expn);
                \end{lstlisting}
                where $(2 +\sim\mathrm{expn})$ is equivalent to $(1+|\mathrm{expn}|)$ in two's complement system. \\\\
                This {\tt +1} is due to shift the implicit leading {\tt 1} at {\tt mprod[104]} to fractional part, since there is no the leading {\tt 1} in subnormal format.

            \subsubsection*{Clock cycle 8}
                This cycle is for rounding. Apply rounding to nearest according to the requirement.
                \begin{lstlisting}[style = verilog-style]
    {mprod[105], mprod[103:52]} <= mprod[103:52] + mprod[51];
                \end{lstlisting}

                The carry is put at {\tt mprod[105]} because we may avoid usage of a logic MUX for {\tt sign\_carry}, which is assigned by

                \begin{lstlisting}[style = verilog-style]
    wire signed [1:0] sign_carry = {1'b0, mprod[105]};
                \end{lstlisting}

            \subsubsection*{Clock cycle 9}
                Cycle 9 is the last cycle in calculation stage.
                We need to generate the result in a correct format.

                \begin{lstlisting}[style = verilog-style]
    if (expn >= sign_0x7FF)
        mprod[103:52] <= 0;
    else if (expn < -52)
        mprod[103:52] <= 0;
                \end{lstlisting}
                \begin{lstlisting}[style = verilog-style]
    if (expn >= sign_0x7FF)
        expn[10:0] <= {11{1'b1}};
    else if (expn > sign_zero)
        expn[10:0] <= expn + sign_carry;
    else if (expn >= -52)
        expn[10:0] <= {{9{1'b0}}, sign_carry};
    else
        expn[10:0] <= 0;
                \end{lstlisting}

                Notice that after the rounding in cycle 8, the {\tt expn} may be added by {\tt 1}. A number in subnormal format may become a number in normal format. 
                Nevertheless, we do not have to deal with it, because the {\tt 1} carried to {\tt mprod[104]} exactly becomes the implicit leading {\tt 1} in normal format.
                
    \section{Design Test}
        In fact, before I started to write the verilog program, 
        I wrote a C program to reach the desired multiplication.
        This helps me to design the algorithm in a more abstract level, without considering some RTL detail.

        With the C program, I can generate arbitrary number of test cases by the C function rand().
        I generated 100000 random patterns to test my verilog implementation.
        However, since the possibilities of appearance of subnormal numbers are too low,
        I set the first 1000 pattern as subnormal numbers by clearing their exponential field to 0. 
        In this way, the testbench can be made very simple. 
        \begin{enumerate}
            \item Generate many input {\tt A} and {\tt B} by C program.
            \item Compute the product by the C program.
            \item Compute the product by my verilog design.
            \item Compare the two products above.
        \end{enumerate}

        In my testbench, the output of my verilog program are actually compared with both results from C and results from Verilog built-in floating multiplication. \\
        The main drawback of my random-generated patterns is that the patterns are very often to become infinity since the exponential field of the input are arbitrary.

        \begin{figure}[h]
            \includegraphics[width=0.7\textwidth]{sim.png}
        \end{figure}

        \begin{itemize}
            \item[$*$] Pattern Location: \verb|/misc/Si2_RAID-1/COURSE/dic/dic03/final_project/Pattern|
            \item[$*$] Testbench Location: \verb|/misc/Si2_RAID-1/COURSE/dic/dic03/final_project/TEST.v|
        \end{itemize}

        \newpage

    \section{Implementation Result}
        
        \subsection{Post-Synthesis Results}
            \noindent pwd: \verb|/misc/Si2_RAID-1/COURSE/dic/dic03/final_project/Lowpower| \\
            netlist: \verb|/misc/Si2_RAID-1/COURSE/dic/dic03/final_project/Lowpower/fp_mult_synLP.v|
            
            \begin{itemize}
                \item Area
                \begin{lstlisting}[backgroundcolor = \color{white},basicstyle=\small\sffamily]
Instance Module  Cell Count  Cell Area  Net Area   Total Area 
--------------------------------------------------------------
fp_mult                4475 127530.850 68126.621   195657.471 
                \end{lstlisting}
                \item Power
                \begin{itemize}
                    \item Total: 0.63 (mW)
                \end{itemize}
                \begin{lstlisting}[backgroundcolor = \color{white},basicstyle=\small\sffamily]
  -------------------------------------------------------------------------
    Category         Leakage     Internal    Switching        Total    Row%
  -------------------------------------------------------------------------
      memory     0.00000e+00  0.00000e+00  0.00000e+00  0.00000e+00   0.00%
    register     1.68908e-06  1.57490e-04  3.45403e-05  1.93720e-04  30.60%
       latch     0.00000e+00  0.00000e+00  0.00000e+00  0.00000e+00   0.00%
       logic     1.30044e-05  2.17370e-04  1.82472e-04  4.12846e-04  65.22%
        bbox     0.00000e+00  0.00000e+00  0.00000e+00  0.00000e+00   0.00%
       clock     7.05780e-08  1.40089e-05  1.23809e-05  2.64604e-05   4.18%
         pad     0.00000e+00  0.00000e+00  0.00000e+00  0.00000e+00   0.00%
          pm     0.00000e+00  0.00000e+00  0.00000e+00  0.00000e+00   0.00%
  -------------------------------------------------------------------------
    Subtotal     1.47641e-05  3.88869e-04  2.29393e-04  6.33026e-04 100.00%
  Percentage           2.33%       61.43%       36.24%      100.00% 100.00%
  -------------------------------------------------------------------------
                \end{lstlisting}
                \item Timing
                \begin{lstlisting}[backgroundcolor = \color{white},basicstyle=\small\sffamily]
Cost Group   : 'CLK' (path_group 'CLK')
Timing slack :    1376ps 
Start-point  : calcount_reg[3]/CK
End-point    : mprod_reg[105]/SI
                \end{lstlisting}
            \end{itemize}

        \subsection{Post-Layout Results}
            \noindent pwd: \verb|/misc/Si2_RAID-1/COURSE/dic/dic03/final_project/APR| \\
            pwd: \verb|/misc/Si2_RAID-1/COURSE/dic/dic03/final_project/DRCLVS| \\
            CHIP.gds: \verb|/misc/Si2_RAID-1/COURSE/dic/dic03/final_project/APR/CHIP.gds| \\
            CHIP.v: \verb|/misc/Si2_RAID-1/COURSE/dic/dic03/final_project/APR/CHIP_postLayout.v| \\
            CHIP\_LVS.v: \verb|/misc/Si2_RAID-1/COURSE/dic/dic03/final_project/APR/CHIP_LVS.v| \\
            CHIP.sdf: \verb|/misc/Si2_RAID-1/COURSE/dic/dic03/final_project/APR/CHIP.sdf|


            \begin{itemize}
                \item Area
                    \begin{itemize}
                        \item CHIP Width: 1130.32 ($\mu$m)
                        \item CHIP Height: 1129.56 ($\mu$m)
                    \end{itemize}
                    \includegraphics[width=0.5\textwidth]{area.png}
                \item Power
                \begin{itemize}
                    \item Total: 2.77 (mW)
                \end{itemize}
                \begin{lstlisting}[backgroundcolor = \color{white},basicstyle=\small\sffamily]
---------------------------------------------------------------------------
Cell                            Internal   Switching       Total     Leakage   Cell
                                   Power       Power       Power       Power   Name
---------------------------------------------------------------------------
---------------------------------------------------------------------------

Total (   4489 of   4531  )        2.194      0.5584       2.77     0.01718
                \end{lstlisting}
                \item Timing
                \begin{lstlisting}[backgroundcolor = \color{white},basicstyle=\small\sffamily]
Analysis View: CHECK_SETUP_TIME
Other End Arrival Time          1.481
- Setup                         0.853
+ Phase Shift                  50.000
+ CPPR Adjustment               0.000
= Required Time                50.628
- Arrival Time                 48.681
= Slack Time                    1.947
                \end{lstlisting}
                \begin{lstlisting}[backgroundcolor = \color{white},basicstyle=\small\sffamily]
Analysis View: CHECK_HOLD_TIME
Other End Arrival Time         25.506
+ Clock Gating Hold             0.000
+ Phase Shift                   0.000
- CPPR Adjustment               0.639
= Required Time                24.867
  Arrival Time                 25.013
  Slack Time                    0.146
                \end{lstlisting}
                \item Simulation Result
                \begin{figure}[h]
                    \includegraphics[width=0.7\textwidth]{postLayout_sim.png}
                \end{figure}
                \item LVS Result
                \begin{figure}[h]
                    \includegraphics[width=0.7\textwidth]{LVS.png}
                \end{figure}
                \item Layout
                \begin{figure}[h]
                    \includegraphics[width=0.8\textwidth]{layout_with_clocktree.png}
                    \centering
                \end{figure}
            \end{itemize}

\end{document}